\documentclass{article}
\usepackage[utf8]{inputenc}
\usepackage{graphicx} % Required for inserting images
\usepackage{wrapfig}  % 图文混排支持
\usepackage{caption}
\usepackage{ctex}
\usepackage{color}
\usepackage{amssymb}
\usepackage{amsmath}
\usepackage{amsthm}
\usepackage[a4paper,margin=1in]{geometry} % 页面设置
\usepackage{wallpaper}

\linespread{2} % 设置行距

\title{Examples in LLM SFT for MATH problems}
\author{Jclian91}
\date{\today}

\begin{document}
	
	\maketitle
	
	\clearpage
	
	\begin{enumerate}
		\item 小明有98元,他要买一本书和一支笔,已知书的价格是25元,笔的价格比书的价格多15元,请问小明还剩余多少元钱?\par
			\includegraphics[width=12cm]{pictures/model_after_1.png}
		\item 抛物线 $y=x^2 - 4x+3$ 的顶点坐标为多少?\par
				\includegraphics[width=12cm]{pictures/model_after_2.png}
		\item 计算123456789与987654321的乘积\par
				\includegraphics[width=12cm]{pictures/model_after_3.png}
		\item 若 $(ax-b)(3x+4)=bx^2 + cx+72$,则$a+b+c$的值为多少?\par
			\includegraphics[width=12cm]{pictures/model_after_4.png}
		\item 123456789 + 987654321 = ?\par
			\includegraphics[width=12cm]{pictures/model_after_5.png}
		\item 已知$\Delta ABC$为正三角形,则 $tan(A+\frac{\pi}{4})$ 的值等于多少?\par
			\includegraphics[width=12cm]{pictures/model_after_6.png}
		\item 各项均为正数的等比数列$\{a_{n}\}$的前n项和为$S_{n}$,已知$S_{3}=10$, $S_{6}=30$,则$S_{12}$等于多少?\par
			\includegraphics[width=12cm]{pictures/model_after_7.png}
		\item 复数 $3+4i$ 的模是多少?\par
			\includegraphics[width=12cm]{pictures/model_after_8.png}
		\item $(1+2x)^6$ 的展开式中 $x^3$ 项的系数为多少?\par
			\includegraphics[width=12cm]{pictures/model_after_9.png}
		\item 学校图书馆有故事书、科技书和连环画共1200本,其中故事书占60\%,科技书和连环画的数量比是2:3,图书馆有多少本连环画?\par
			\includegraphics[width=12cm]{pictures/model_after_10.png}
		\item 某天,小明去菜市场买了5个苹果和3个梨,花了20元。第二天,他又去同一个菜市场买了3个苹果和4个梨,花了15元。问:小明每个苹果和每个梨的价格是多少元?\par
			\includegraphics[width=12cm]{pictures/model_after_11.png}
		\item Suppose you have a system of linear equations: 
		\begin{equation*}
			\begin{aligned}
				2x + 3y + z & = 7 \\
				x + 2y + 4z & = 12 \\
				3x + y + 2z & = 8 
			\end{aligned}
		\end{equation*}
		Using matrix methods, find the values of x, y, and z that satisfy the system of equations.\par
			\includegraphics[width=12cm]{pictures/model_after_12.png}
		\item $\int_{0}^{\pi^2} \cos(\sqrt{x}) dx $\par
			\includegraphics[width=12cm]{pictures/model_after_13.png}
	\end{enumerate}

\end{document}